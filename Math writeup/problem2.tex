\documentclass[11pt, oneside]{article}   	% use "amsart" instead of "article" for AMSLaTeX format

\usepackage{geometry}                		% See geometry.pdf to learn the layout options. There are lots.
\geometry{letterpaper}                   		% ... or a4paper or a5paper or ... 
%\geometry{landscape}                		% Activate for rotated page geometry
\usepackage[parfill]{parskip}    		% Activate to begin paragraphs with an empty line rather than an indent
\usepackage{graphicx}				% Use pdf, png, jpg, or eps§ with pdflatex; use eps in DVI mode
								% TeX will automatically convert eps --> pdf in pdflatex
\usepackage{amsmath}


\title{The cable project}
\author{Elise Niedringhaus, Sarah Liddle and Darice Guittet}
%\date{}							% Activate to display a given date or no date

\begin{document}
\maketitle
\section{2}
\subsection{2a}
\subsection{2b}
Boundary conditions:\\
$$\lim_{x\to\infty} \frac{\partial v(x,t)}{\partial x}=0$$ and $$\lim_{x\to\infty} \frac{\partial v(x,t)}{\partial x}=0$$\\
Application of the change of variables $v(x,t)=V(\xi)$, where $\xi=x-ct$ to $\frac{\partial v(x,t)}{\partial t}=\frac{\partial ^2 v(x,t)}{\partial x^2}-v(x,t)+H(v-\theta)+J_{ext}(x,t)$ yields the following second order ordinary differential equation. Assume that the external current. $J_{ext})(x,t)=0$.\\
\begin{equation}
\label{second order ODE}
-cV'(\xi)=V''(\xi)-V(\xi)+H(V(\xi)-\theta)
\end{equation}
\subsection{2c}

The two ordinary differential equations are:\\
\begin{equation}
\label{ODE1}
-cV_1'(\xi)=V_1''(\xi)-V_1(\xi)
\end{equation}
for  $\xi \in (0,\infty)$ and\\

\begin{equation}
\label{ODE2} 
-cV_2'(\xi)=V_2''(\xi)-V_2(\xi)+1
\end{equation}
for $\xi \in (-\infty,0)$

The boundary conditions are\\
$$\lim_{\xi\to-\infty} \frac{d V_1(\xi)}{d \xi}=0$$
$$\lim_{\xi\to\infty} \frac{d V_2(\xi)}{d \xi}=0$$
$$\lim_{\xi\to-\infty} V_1(\xi)=1$$
$$\lim_{\xi\to\infty} V_2(\xi)=0$$
$$V_1(0)=V_2(0)$$
$$\frac{d V_1}{d \xi}(0)=\frac{d V_2}{d \xi}(0)$$
\subsection{2d}

Equation \ref{ODE1} is a homogeneous ordinary differential equation. The corresponding characteristic equation, $r^2+cr-1=0$, has roots of  $r=\frac{-c \pm \sqrt{c^2+4}}{2}$. Thus, 
\begin{equation}
\label{ODE1 general solution}
V_{1}=c_1 e^{\frac{1}{2}(-c+\sqrt{c^2+4})\xi}+c_2e^{-\frac{1}{2}(c+\sqrt{c^2+4})\xi}, \xi \in (0,\infty)
\end{equation}

In order to solve equation \ref{ODE2}, use the method of undetermined coefficients. First, determine the homogeneous solution of the differential equation by solving $cV_2'(\xi)=V_2''(\xi)-V_2(\xi)$ for $V_2$. The characteristic equation, $r^2+cr-1=0$, has roots of $r=\frac{-c \pm \sqrt{c^2+4}}{2}$. Thus, the homogeneous solution is
\begin{equation}
\label{homogeneous solution}
V_{2,h}=c_3 e^{\frac{1}{2}(-c+\sqrt{c^2+4})\xi}+c_4 e^{-\frac{1}{2}(c+\sqrt{c^2+4})\xi}
\end{equation}
Guess a particular solution of the form $V_{2,p}=A$. Plugging $V_{2,p}$ into equation \ref{ODE2} yields 
$$-c(0)=0-A+1$$
$$A=1$$ 
$$V_{2,p}=1$$
Thus, the solution to equation \ref{ODE2} is 
\begin{equation}
\label{ODE2 general solution}
V_{2}=c_3 e^{\frac{1}{2}(-c+\sqrt{c^2+4})\xi}+c_4 e^{-\frac{1}{2}(c+\sqrt{c^2+4})\xi}+1, \xi \in (-\infty,0)
\end{equation}
Next, use the boundary conditions to eliminate the arbitrary coefficients. 
$$\lim_{\xi\to\infty} \frac{d V_1(\xi)}{d \xi}=0\Rightarrow c_1=0$$
$$\lim_{\xi\to-\infty} \frac{d V_2(\xi)}{d \xi}=0\Rightarrow c_4=0$$
$$V_1(0)=V_2(0)\Rightarrow c_2=c_3+1$$
$$\frac{d V_1}{d \xi}(0)=\frac{d V_2}{d \xi}(0)\Rightarrow -\frac{1}{2}(c+\sqrt{c^2+4}) c_2= \frac{1}{2}(-c+\sqrt{c^2+4})c_3$$ 
The solution to this linear system of equations is $c_1=0, c_2=\frac{\frac{1}{2}(-c+\sqrt{c^2+4})}{\frac{1}{2}(-c+\sqrt{c^2+4})+\frac{1}{2}(c+\sqrt{c^2+4})}=\frac{-c+\sqrt{c^2+4}}{2\sqrt{c^2+4}}, c_3= \frac{-\frac{1}{2}(c+\sqrt{c^2+4})}{\frac{1}{2}(-c+\sqrt{c^2+4})+\frac{1}{2}(c+\sqrt{c^2+4})}=\frac{-c-\sqrt{c^2+4}}{2\sqrt{c^2+4}}$, and $c_4=0$.\\
The solution to equation \ref{ODE1} is 
\begin{equation}
\label{ODE1 solution}
V_1(\xi)=\frac{-c+\sqrt{c^2+4}}{2\sqrt{c^2+4}}e^{-\frac{1}{2}(c+\sqrt{c^2+4})\xi},\xi \in (0,\infty).
\end{equation}
The solution to equation \ref{ODE2} is
\begin{equation}
\label{ODE2 solution}
V_2(\xi)=\frac{-c-\sqrt{c^2+4}}{2\sqrt{c^2+4}}e^{\frac{1}{2}(-c+\sqrt{c^2+4})\xi}+1,\xi \in (-\infty,0).
\end{equation}
\subsection{2e}
In order to understand the relationship between the speed of the traveling front and the threshold value $\theta$, apply the threshold condition that $V(0)=\theta$. Application of this threshold condition to the traveling wave solutions gives $\frac{-c}{2\sqrt{c^2+4}}+\frac{1}{2}=\theta$.
$$\frac{-c}{\sqrt{c^2+4}}+1=2\theta$$
$$\frac{-c}{\sqrt{c^2+4}}=2\theta1$$
$$\frac{c^2}{c^2+4}=(2\theta-1)^2$$
$$c^2(1-(2\theta-1)^2)=4(2\theta-1)^2$$
$$c=\sqrt{\frac{4(2\theta-1)^2}{1-(2\theta-1)^2}}$$
$$c=\sqrt{\frac{-(2\theta-1)^2}{\theta^2-\theta}}$$
\subsection{2f}
Partial differential equation: 
\begin{equation}
\label{uh}
\frac{\partial v}{\partial t}=\frac{\partial ^2 v}{\partial x^2}+f(v(x,t))
\end{equation}
where $f(v)=-v+H(v-\theta)$.\\
In order to analyze the stability of the traveling wave solution found, let $v(x,t)=V(\xi)+\epsilon\psi(\xi,t)$, where $0 < \epsilon <<1$ and $\psi(\xi,t)$ represents a small perturbation to the traveling wave solution. 
$$\frac{\partial v}{\partial t}=\frac{\partial\xi}{\partial t}\frac{dV}{d\xi}+\epsilon(\frac{\partial \psi}{\partial \xi}\frac{\partial \xi}{\partial t}+\frac{\partial \psi}{\partial t})$$
$$\Rightarrow \frac{\partial v}{\partial t}=-c\frac{dV}{d\xi}+\epsilon(-c\frac{\partial \psi}{\partial \xi}+\frac{\partial \psi}{\partial t})$$
$$\frac{\partial v}{\partial x}=\frac{\partial \xi}{\partial x}\frac{dV}{d\xi}+\epsilon\frac{\partial\xi}{\partial x}\frac{\partial\psi}{\partial\xi}$$
$$\Rightarrow \frac{\partial v}{\partial x}=\frac{dV}{d\xi}+\epsilon\frac{\partial\psi}{\partial\xi}$$
$$\frac{\partial^2 v}{\partial x^2}=\frac{\partial \xi}{\partial x}\frac{d^2V}{d\xi^2}+\epsilon\frac{\partial^2\psi}{\partial\xi^2}\frac{\partial \xi}{\partial x}$$
$$\Rightarrow \frac{\partial^2 v}{\partial x^2}=\frac{d^2V}{d\xi^2}+\epsilon\frac{\partial^2\psi}{\partial\xi^2}$$
Equation \ref{uh} can be rewritten as 
\begin{equation}
\label{frewritten}
-c\frac{dV(\xi)}{d\xi}+\epsilon(-c\frac{\partial \psi(\xi,t)}{\partial \xi}+\frac{\partial \psi(\xi,t)}{\partial t})=\frac{d^2V(\xi)}{d\xi^2}+\epsilon\frac{\partial^2\psi(\xi,t)}{\partial\xi^2}+f(V(\xi)+\epsilon\psi(\xi,t))
\end{equation}
where $f(V(\xi)+\epsilon\psi)=-(V(\xi)+\epsilon\psi(\xi,t))+H(V(\xi)+\epsilon\psi(\xi,t)-\theta)$.\\

Analysis of equation \ref{frewritten} can be simplified by using the Taylor expansion of $f(V(\xi)+\epsilon\psi)$ with respect to $\epsilon$, about $\epsilon=0$.
$$f(V(\xi)+\epsilon\psi)\approx f(V(\xi)+\epsilon\psi)\Big|_{\epsilon=0}+\Bigg(\frac{\partial f(V(\xi)+\epsilon\psi)}{\partial \epsilon}\Big|_{\epsilon=0}\Bigg)\epsilon+O(\epsilon ^2)$$
$$f(V(\xi)+\epsilon\psi)\approx f\big(V(\xi)\big)+\big(-\psi(\xi,t)+\psi(\xi,t) \delta(V(\xi)-\theta)\big)\epsilon$$
$$f(V(\xi)+\epsilon\psi)\approx -V(\xi)+H(V(\xi)-\theta)+\psi(\xi,t)\epsilon\big(-1+\delta(V(\xi)-\theta)\big)$$
Equation \ref{frewritten} becomes
$$-c\frac{dV(\xi)}{d\xi}+\epsilon(-c\frac{\partial \psi(\xi,t)}{\partial \xi}+\frac{\partial \psi(\xi,t)}{\partial t})=\frac{d^2V(\xi)}{d\xi^2}+\epsilon\frac{\partial^2\psi(\xi,t)}{\partial\xi^2}-V(\xi)+H(V(\xi)-\theta)+\psi(\xi,t)\epsilon\big(-1+\delta(V(\xi)-\theta)\big)$$
In order to analyze the long term behavior of the disturbance, look only at the terms involving $\psi$ and $\epsilon$. The partial differential equation that models the behavior of the perturbation is:
$$\epsilon(-c\frac{\partial \psi(\xi,t)}{\partial \xi}+\frac{\partial \psi(\xi,t)}{\partial t})=\epsilon\frac{\partial^2\psi(\xi,t)}{\partial\xi^2}+\psi(\xi,t)\epsilon\big(-1+\delta(V(\xi)-\theta)\big)$$
$$-c\frac{\partial \psi(\xi,t)}{\partial \xi}+\frac{\partial \psi(\xi,t)}{\partial t}=\frac{\partial^2\psi(\xi,t)}{\partial\xi^2}+\psi(\xi,t)\big(-1+\delta(V(\xi)-\theta)\big)$$
In order to further understand the behavior of the perturbation, solve the partial differential equation to the left and the right of $V(\epsilon)=\theta$. The governing equation for $V(\xi)\neq\theta$ is 
\begin{equation}
\label{psiPDE}
-c\frac{\partial \psi(\xi,t)}{\partial \xi}+\frac{\partial \psi(\xi,t)}{\partial t}=\frac{\partial^2\psi(\xi,t)}{\partial\xi^2}-\psi(\xi,t)
\end{equation}
since $\delta(V(\xi)-\theta)=0$ whenever $V(\xi)\neq\theta$. This equation can be solved using separation of variables, which means solutions should have the form of $\psi(\xi,t)=F(\xi)G(t)\neq0$.
This proposed solution must satisfy the partial differential equation in equation \ref{psiPDE}.
$$-cG(t)\frac{dF(\xi)}{d\xi}+F(\xi)\frac{dG(t)}{dt}=G(t)\frac{d^2F(\xi)}{d\xi^2}-F(\xi)G(t)$$
$$-c\frac{1}{F(\xi)}\frac{dF}{d\xi}+\frac{1}{G(t)}\frac{dG}{dt}=\frac{1}{F(\xi)}\frac{d^2F(\xi)}{D\xi^2}-1$$
$$$$




\end{document}  