\documentclass[12pt]{article}
\usepackage{graphicx}
\usepackage{amssymb}
\graphicspath{ {images/} }
\usepackage[margin=1in]{geometry}
\usepackage{float}
\title{Cable Project}
\author{Darice, Elise, Sarah}


\begin{document}
\maketitle
\section{Perturbation}
\subsection{PDE for perturbation function}

\begin{equation} \label{pde_nh}
\frac{\partial \psi}{\partial t} = c \frac{\partial^2\psi}{\partial\xi^2} + \frac{\partial\psi}{\partial\xi} - (1 - \delta(V(\xi)-\theta))\psi
\end{equation} 

We split the domain of the equation into three regions, depending on the value of $\delta(V(\xi)-\theta)$, with the boundary condition that $\lim_{\xi \to \pm \infty}\psi = 0$:

1) $\xi < 0$ and $\delta(V(\xi)-\theta) = 0$

2) $\xi > 0$ and $\delta(V(\xi)-\theta) = 0$ 

3) $\xi = 0$ and $\delta(V(\xi)-\theta) = \infty$


\subsection{Separation of Variables}
Looking first at region 1 and 2, we have the homogeneous PDE.

\begin{equation} \label{pde_h}
\frac{\partial \psi}{dt} = c \frac{\partial^2\psi}{d\xi^2} + \frac{\partial\psi}{\partial\xi} - \psi 
\end{equation}
We assume a separation of variables, $\psi(\xi,t) = S(\xi)T(t)$, which leads to

$$ \frac{1}{T}\frac{\partial T}{\partial t} = \frac{1}{S}\frac{\partial^2S}{\partial\xi^2} + \frac{c}{S}\frac{\partial S}{\partial\xi} - 1 = \lambda $$
The general solution is

$$ S(\xi) = c_1e^{\frac{-c+\sqrt{c^2+4(\lambda+1)}}{2}\xi} + c_2e^{\frac{-c-\sqrt{c^2+4(\lambda+1)}}{2}\xi} $$

\subsection{Region 1 and 2}
In Region 1, where $\lim_{\xi \to -\infty}S_1 = 0$, and in Region 2, where $\lim_{\xi \to \infty}S_2 = 0$, we have the solutions:

\begin{equation} \label{homogsln}
S_1(\xi) = c_1e^{\frac{-c+\sqrt{c^2+4(\lambda+1)}}{2}\xi}
S_2(\xi) = c_2e^{\frac{-c-\sqrt{c^2+4(\lambda+1)}}{2}\xi} 
\end{equation}

\subsection{Region 3}
At Region 3, we integrate the equation at a small interval around the point of discontinuity to examine the effect $\delta(V(\xi)-\theta)$ has on the solution. After separation of variables, the inhomogeneous PDE \ref{pde_nh} becomes

$$ \frac{1}{T}\frac{\partial T}{\partial t} = \frac{1}{S}\frac{\partial^2S}{\partial\xi^2} + \frac{c}{S}\frac{\partial S}{\partial\xi} - 1 + \delta(V(\xi)-\theta)= \lambda $$
The spatial equation is 

\begin{equation}\label{spatialdelta}
\frac{d^2S}{d\xi^2} + c\frac{dS}{d\xi} - [\lambda + 1 + \delta(V(\xi)-\theta)]S = 0
\end{equation} 
Since there is a discontinuity due to the dirac delta function here at $V(\xi) = \theta$, $\xi = 0$, we integrate a small interval $(+\epsilon,-\epsilon)$ around the discontinuity. Then, we taking the limit $\epsilon \to 0$ gives us information as to how the solutions in Region 1 and 2 (\ref{homogsln}) connect in Region 3.

$$ \int_{-\epsilon}^{+\epsilon}\frac{d^2S}{d\xi^2}d\xi + c\int_{-\epsilon}^{+\epsilon}\frac{dS}{d\xi}d\xi - \int_{-\epsilon}^{+\epsilon}(\lambda + 1)Sd\xi - \int_{-\epsilon}^{+\epsilon}\delta(V(\xi)-\theta)Sd\xi = 0 $$

For the first two terms, the fundamental theorem of calculus gives us the first derivative evaluated at the endpoints and the function evaluated at the endpoints.

$$ \frac{dS}{d\xi}(+\epsilon) - \frac{dS}{d\xi}(-\epsilon) + c[S(+\epsilon) - S(-\epsilon)] - \int_{-\epsilon}^{+\epsilon}(\lambda + 1)Sd\xi - \int_{-\epsilon}^{+\epsilon}\delta(V(\xi)-\theta)Sd\xi = 0 $$
Now we take the limit:
$$ \lim_{\xi \to 0} \Bigg[ \frac{dS}{d\xi}(+\epsilon) - \frac{dS}{d\xi}(-\epsilon) + c[S(+\epsilon) - S(-\epsilon)] - \int_{-\epsilon}^{+\epsilon}(\lambda + 1)Sd\xi - \int_{-\epsilon}^{+\epsilon}\delta(V(\xi)-\theta)Sd\xi \Bigg]= 0 $$
Of the remaining two integrals, the first one goes to zero as the bounds shrink the zero since the integral is continuous, even if $S$ is not. The second integral involving the dirac delta function evaluates to $S(0)$ by the \textit{sampling property}, since we are specifically looking at the small region centered around where $V(\xi) = \theta$. Therefore we get:
$$ \lim_{\xi \to 0} \Bigg[ \frac{dS}{d\xi}(+\epsilon) - \frac{dS}{d\xi}(-\epsilon) + c[S(+\epsilon) - S(-\epsilon) \Bigg] = S(0) $$
$S$ must be continuous if we are assuming that the derivative $\frac{dS}{d\xi}$ exists so $S(0^+) = S(0^-)$ and the constants in front of the homogeneous solutions (\ref{homogsln}) are equal, $c_1 = c_2 = K$, revealing a relationship between the value of the spatial term and the discontinuity between its derivative at $\xi=0$.
$$ S(0) = \lim_{\xi \to 0} \Bigg[ K(\frac{-c-\sqrt{c^2+4(\lambda+1)}}{2}e^{\frac{-c-\sqrt{c^2+4(\lambda+1)}}{2}} - \frac{-c+\sqrt{c^2+4(\lambda+1)}}{2}e^{\frac{-c+\sqrt{c^2+4(\lambda+1)}}{2}})\Bigg] $$
Taking the limit, we find the following, where $S(0)\in \mathbb{R}$ and $K\in\mathbb{R}$ since this is a physical system and we assume the solutions are real-valued.
\begin{equation}\label{Keq}
S(0) = K\Bigg(\frac{-c-\sqrt{c^2+4(\lambda+1)}}{2} - \frac{-c+\sqrt{c^2+4(\lambda+1)}}{2} \Bigg)
\end{equation}
$S$ is the spatial part of some arbitrary perturbation function that satisfies $\lim_{\xi \to \pm \infty}S(\xi) = 0$, and its value at zero could be either non-zero, or it could be 0. Since the boundary conditions provided do not fix $S(0)$ or $K$, we will examine both cases. In the latter case, if $S(0) = 0$ then $K=c_1=0$ and we find the trivial solution of $S=0$ which gives us $\psi(\xi,t) = 0$. In the following section, we explore the former case.


\subsection{Eigenvalues}
Now to examine the family of solutions that depend on $\lambda$, we start with the $\lambda = 0$ case. Assuming we can rescale $S(0)$ to unity, eq (\ref{Keq}) simplifies to:
$$1 = K\Bigg(\frac{-c-\sqrt{c^2+4)}}{2} - \frac{-c+\sqrt{c^2+4)}}{2}\Bigg) = -K\sqrt{c^2+4} $$
or
$$K = -\frac{1}{\sqrt{c^2+4}}$$
Plugging this constant back into our 


\end{document}
